\chapter{I18n - Internationalisierung}
i18n steht f�r Internationalisierung (z�hle die Anzahl der Buchstaben zwischen i und n ;). Seit 0.9.8 werden auch mehrere Sprachen unterst�tzt. Wir werden uns weiterhin bem�hen XChat zu internationalisieren. Leider sind im Moment nur die Men�s den Sprachen angepasst. Um Deine gew�nschte Sprache auszuprobieren, musst Du folgendes tun:
\begin{quote}
\texttt{export LANG=xx}
\end{quote}
wobei \texttt{xx} der gew�nschte Sprachencode ist. Solltest Du als Shell etwas anderes als \texttt{bash} benutzen, musst Du nat�rlich den Syntax ver�ndern.
\chapter{Autoren}
\section{Autoren der englischen Dokumentation}
Viele, viele Leute haben XChat geholfen. Zu viel, um diese hier aufzulisten. Ihr wisst, wen wir meinen. Danke an Euch.
\begin{itemize}
  \item Peter Zelezny \texttt{zed@linuxpower.org} (Das meiste vom XChat)
  \item Erik Scrafford \texttt{eriks@chilisoft.com} (perl.c, lastlog.c, color.c)
  \item Adam Langley \texttt{agl@linuxpower.org} (plugins, diese Documentation, TextEvents, ...)
  \item Dagmar d'Surreal \texttt{nospam@dsurreal.org} (rfc1459 Zeichenkettenvergleich util.c, siehe auch Kommentare)
  \item Matthias Urlichs \texttt{smurf@noris.de} (Perl text events)
  \item David Herdeamn \texttt{david@2gen.com} (Ignore GUI, Baum Serverliste, Untermen�s in Popups)
  \item Scott James Remnant \texttt{scott@netsplit.com} (Highlight notifies, Prefs GUI, IP settings)
\end{itemize}

Viele andere haben mit sonstigen Ver�nderungen geholfen. Solltest Du einen Patch �bermittelt haben und m�chtest, dass Dein Name hier erscheint, lass es \textbf{zed@linuxpower.org} wissen.

\subsection{Maintainers} 
Peter Zelezny (alias: zed) f�gt alle Patches zu einem
(hoffentlich richtigen) etwas zusammen. Er ist f�r die Website zust�ndig und
kontrolliert alle ``wirklichen'' Ver�ffentlichungen vom XChat, welche von ihm
kommen. Er verwaltet auch die Freshmeat und GNOME AppList Bekanntmachungen.
Jede Ver�nderung im ChangeLog ohne Namen ist sein Werk. Seine E-Mail Adresse
ist: \textbf{zed@linuxpower.org}.


Adam Langley (alias: Nebulae) verwaltet die Dokumentation und einige Brocken
von Code, meistens Signal- und den Plugin Code. Ver�ffentlichungen von ihm
sind meistens nicht die ``wirklichen'' Ver�ffentlichungen und meistens nicht
stabil.


Andere Leute, die Code und Ideen mit in das Projekt bringen, gehen meistens an
zed - im Elitenet - (\#linux).


Patches sollten zu Peter gemailt werden. Solltest Du Ratschl�ge oder Hilfe f�r
den XChat gebrauchen, schau Dir als erstes dieses Dokument an, dann frage
jemanden im Elitenet\footnote{Server: irc.xchat.org} (\#linux), aber denke
daran, dass die Leute in \#linux nicht irgendwelchen Schrott unterst�tzen. Sie
werden �ber Dich lachen.

\section{Autoren der deutschen Dokumentation}
\label{sec:Autoren}
Hier seien nur die aufgez�hlt, die bei der �bersetzung ins Deutsche mitgewirkt haben. Vielen Dank nochmal an jene die zur Verbesserung des Dokumentes beigetragen haben.
\begin{itemize}
  \item Roman Joost \texttt{romanjoost@gmx.de} - http://www.romanofski.de (�bersetzung der englischen Dokumentation ins Deutsche)
  \item Marika Wolff \texttt{mari\_wo@web.de} (Korrigieren der vielen Fehler)
  \item Rolf Eike Beer \texttt{eike@bilbo.math.uni-mannheim.de} (Korrigieren
    von Fehlern)
\end{itemize}

\section{Einschicken von korrigiertem Text}

F�r alle, die uns helfen wollen, hier eine kurze Anleitung wie man korrigierte
Texte erstellt. Bitte denkt daran, dass die Dokumentation sehr gro� ist und wir
nicht sehr viel Zeit haben,um uns ellenlange Text durchzulesen, worin vermerkt
ist, dass in Zeilennummer ``sowieso'' ein Fehler verborgen ist. \textbf{Bitte
  denkt daran, dass ihr euch f�r die entsprechenden Sprachen an die
  entsprechenden Autoren wendet.} Im Grunde genommen ist es ganz einfach: 

\begin{enumerate}
\item Ihr nehmt die Originaldatei (*.tex) und korrigiert die Textbereiche die Fehler enthalten bzw. f�gt Text hinzu, wo Ihr denkt, das dort noch was fehlt. Solltet Ihr die Originaldatei nicht zu diesem Dokument erhalten haben, ladet sie euch einfach unter \textbf{www.xchat.org} oder \textbf{www.romanofski.de} herunter.
\item H�ngt das ganze Dokument an eine E-Mail und schickt es an einen von uns (E-Mail stehen bei \textbf{\ref{sec:Autoren}}
\item Wir k�mmern uns um den Rest. 
\end{enumerate}

\chapter{�bersicht der Tastaturk�rzel im XChat 2}

\begin{longtable}{|p{3cm}|p{7cm}|}
\hline
\textbf{Tastaturk�rzel} & \textbf{Funktion}\\
\hline
\hline
Strg + S & Server List\\
Strg + I & Detach Tab \\
Shift+Strg+W & Close Tab\\
Strg+Q & Beenden\\
Alt+A & Als abwesend markieren\\
Strg+L & Clear Text\\
Strg+F & Search Text\\
\hline
\end{longtable}

