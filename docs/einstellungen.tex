%next chapter
\chapter{Einstellungen}
\section{Oberfl�che}
\begin{itemize}
\item Keine Serverliste beim Start - Wenn das gesetzt ist, wird beim Programmstart keine Serverliste angezeigt.
\item URL-Liste automatisch speichern - Speichert die URL-Liste beim Beenden.
\item Doppelklick-Kommando - Das Kommando wird ausgef�hrt, wenn man auf einen Benutzer in der Benutzerliste doppelt klickt. \%s in der Option wird mit dem Spitznamen ersetzt, bevor es ausgef�hrt wird.
\end{itemize}

\subsection{IRC Eingabe/Ausgabe}
\begin{description}
\item[Vervollst�ndigen der Spitznamen] - Durch Setzen wird der eingegebene Text nach falschen Spitznamen durchsucht und berichtigt. Siehe auch Tab Spitznamen.
\item[Zeitmarkierung f�r gesamten Text] - Hier wird vor jeder neuen Zeile die Uhrzeit mit ausgedruckt.
\item[Tab-Nicks] - Spitznamen und Tedt werden mit einem Tabulator angeordnet.
\item[Farbige Spitznamen] - Jetzt werden Spitznamen farbig angezeigt.
\item[BEEPs ausfiltern] - Jetzt werden alle BEEP-Codes ausgefiltert.
\item[Textpuffer-Gr��e] - Die Nummer von Zeilen, welche gepuffert werden (0 = alle Zeilen).
\item[Einladungen im aktiven Fenster anzeigen] - Sollte Euch ein Benutzer in ein Kanal einladen, dann seht Ihr das im aktiven Fenster.
\item[MIRC-Farben entfernen] - Farben werden nicht mit angezeigt, wenn das angeklickt ist.
\end{description}

\subsection{Fensterlayout}\label{fensterlayout}
\begin{description}
\item[Kanalmodus-Kn�pfe] - Wenn das angeklickt ist, werden die Modi-Kn�pfe in der Werkzeugleiste angezeigt.
\item[Benutzerlisten-Kn�pfe] - Wenn das angeklickt ist, werden die Kommando-Kn�pfe unter der Benutzerliste angezeigt.
\item[Lag meter und Throttle meter:] - Hier kann man einstellen, wie die Ausgabe des \textbf{Lag meters} ist - Text oder grafisch als Bar.
Diese Indikatoren geben Dir �ber die Verbindung zum Server Auskunft.
\item[Neue Reiter nach vorne] - Hier werden neue Kanalreiter nach vorne gebracht.
\item[Kanal-Reiter] - Reiter anstatt neue Fenster benutzen.
\item[Private Nachrichten-Reiter] - Hier werden private Nachrichten in Reitern angezeigt.
\item[Reiterbefinden sich:] - Reiter werden am unteren Ende des Fensters angezeigt.
\item[Use a separate tab/window for server messages] - Server-Nachrichten werden in einem Kanal ausgegeben und je nach Einstellung in dem eigenen oder in einem seperaten Reiter.
\item[Disable Paned Userlist] - When this is set the userlist cannot be paned across to be made bigger and is of fixed size \textbf[FIXME]
\end{description}

\subsection{Hauptfenster}
\begin{itemize}
\item Links und Oben beschreiben die Position des Fensters beim Start dar
\item Breite und H�he setzen die Gr��e des XChat Fensters
\end{itemize}

\subsection{Kanalfenster \& Dialogfenster}
Diese 2 Punkte sind eigentlich dasselbe, bis auf das, was in ihnen passiert:
\begin{description}
\item[Schriftart] - Die Schrift, die f�r den Standardtext benutzt wird.
\item[Fettschrift] - Die Schrift, die f�r Fettschrift benutzt wird.
\item[Hintergrundbild] - Ein Bild, das im Hintergrund des Textk�stchens angezeigt wird.
\item[Transparenter Hintergrund] - Der Hintergrund ist pseudo-transparent.
\item[Transparenz einf�rben] - Die Transparenz wird mit einem bestimmten Farbton eingef�rbt.
\end{description}

\section{IRC}
\begin{description}
\item[Rohe Modusanzeige] - Wenn das gesetzt ist, werden die rohen Modi als beschreibende Texte im IRC angezeigt.
\item[Bei privaten Nachrichten piepsen] - Wenn das angeklickt ist, wird der PC-Lautsprecher dazu benutzt, um private Nachrichten anzuzeigen.
\item[Beendigungs-Nachricht] - Der Text, der als Grund des Beendens angezeigt wird.
\item[DNS Lookup Programm] - Name des Programms, welches f�r das Aufsuchen der IPs benutzt wird.
\item[Auto Reconnect-Verz�gerung] - Anzahl der Sekunden zu warten, bevor wieder zum Server verbunden wird.
\end{description}

\subsection{IP Addresse}
\begin{description}
\item[Autodetect hostname] - Hier wird versucht, den Hostnamen automatisch zu ermitteln.
\item[Autodetect IP adress] - Wenn das gesetzt ist, wird die lokale IP-Adresse ermittelt.
\item[Hostname] - Wenn \textbf{automatisch ermitteln} nicht eingestellt ist, wird das als Hostname verwendet.
\item[IP Adresse] - Wenn \textbf{automatisch ermitteln} nicht eingestellt ist, ist dies die IP-Adresse.
\item[IP vom Server holen] - (Nur wenn \textbf{automatisch ermitteln} eingestellt ist) Bezieht die IP Adresse vom Server.
\end{description}

\subsection{Proxy Server}
\begin{description}
\item[Hostname des Proxy Servers] - Der Hostname des Proxy Servers, z.B. mein.proxyserver.de
\item[Port Nummer des Proxy Servers] - Eine Portnummer die zw. 0 - 65535 liegen darf.
\item[Proxy Typ] - Man kann zwischen Wingate, Socks4, Socks5 und einem HTTP Proxy ausw�hlen.
\end{description}

\subsection{Abwesend}
\begin{description}
\item[Abwesenheit einmal zeigen] - Wenn das eingestellt ist, wird der Abwesenheitsgrund nur einmal angezeigt.
\item[Abwesenheits-Meldung ank�ndigen] - Wenn das eingestellt ist, wird der Abwesenheitsgrund gebroadcastet.
\item[Abwesenheitsgrund] - Der Standard Abwesenheitsgrund.
\end{description}

\subsection{Markieren}
\begin{description}
\item[Zu markierende W�rter] - W�rter (wie Dein Spitzname) die markiert werden, wenn Sie im Text vorkommen.
\end{description}

\subsection{Logb�cher}
\begin{description}
\item[Logb�cher] - Wenn das eingeschaltet ist, werden die Logb�cher im Verzeichnis \texttt{~/.xchat/xchatlogs} abgelegt.
\item[Logb�cher immer mit Zeitstempel versehen] - Die Logb�cher werden nach Einstellung entweder mit oder ohne Zeitstempel versehen.
\item[Maske f�r Logb�cher] - Hier stellt man die Maske ein, in welchem Format die Logb�cher abgelegt werden.
\item[Log timestamp format:] - Das Format, wie die Uhrzeit geschrieben wird.
\end{description}

\subsection{Notification}
\begin{description}
\item[Notifies markieren] - Wenn das eingestellt ist, werden die Spitznamen in der Benutzerliste farbig gezeigt, wenn diese in der Benachrichtigungsliste auftauchen.
\item[Farbe f�r Benutzer mit Notify] - Die Farbe f�r das oben Besprochene benutzen.
\item[Notify - �berpr�fungsintervall] - Die Anzahl von Sekunden, in dem der Status der Leute abgefragt wird (0 - nicht �berpr�fen).
\end{description}

\subsection{Zeichensatz}
Hier k�nnen die im ircII benutzten Zeichen�bersetzungstabellen geladen werden.

\subsection{CTCP}
\begin{description}
\item[Version unkenntlich machen] - Wenn das eingestellt ist, wird die Versionsanfrage von anderen ignoriert.
\item[Soundverzeichnis] - Das Verzeichnis, in dem nach Sounds gesucht wird.
\item[Abspielkommando] - Das Kommando wird benutzt, um Sounds abzuspielen.
\end{description}

\section{DCC}
\begin{description}
\item[Auto*] - Hier wird eingestellt, ob die entsprechenden Fenster automatisch ge�ffnet werden sollen.
\end{description}

\subsection{Dateitransfer}
\begin{description}
\item[DCC bietet Timeout an:] Die Anzahl der Sekunden, bis das DCC Angebot entfernt wird (0  = ausschalten).
\item[DCC-Abbruch-Zeitschwelle:] - Die Anzahl der Sekunden, bis eine abgebrochene Verbindung beendet wird (0 = ausschalten).
\item[Dateiberechtigungen] - Die Dateiberechtigungen in Oktal f�r die abzuspeichernden Dateien (0600 wird empfohlen)
\item[Verzeichnis zum Abspeichern] - Das Verzeichnis in dem die DCC Dateien abgelegt werden.
\item[Datei mit Spitznamen abspeichern] - Im Namen der abgespeicherten Datei, wird der Spitzname des Senders mit vermerkt.
\item[Schnelles DCC-Senden] - Wenn das eingestellt ist, wartet DCC nicht auf Best�tigungen, bevor ein n�chstes Paket gesendet wird (Fehler k�nnen aber somit nicht �berpr�ft werden).
\end{description}

