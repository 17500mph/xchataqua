\chapter{Wie kann man XChat helfen ?}
\section{Navigieren im Code}
Die Hauptquellen vom XChat befinden sich in dem /src Verzeichnis. Darin sind alle *.c und *.h Dateien, welche XChat ausmachen.
Solltest Du Dich ein bisschen im Code umschauen wollen, ist hier eine kleine Karte:
\begin{itemize}
\item xchat.c - Hauptprogrammdatei, beinhaltet main()
\item xchat.h - Hauptbibliothekendatei, welche die meisten Hauptstrukturen im XChat benutzt
\item editlist.c - normaler Code, der zum Behandeln von editierbaren Listen benutzt wird (z.B. Liste der Benutzerlistenkn�pfe)
\item fkeys.c - behandelt die Funktionstasten
\item gtkutil.c - wrappt GTK
\item outbound.c - Code f�r die Kommandobehandlung
\item inbound.c - Code f�r die Datenbehandlung vom Server
\item text.c - Code f�r die Textbehandlung und das Logging
\item plugin.c - der ganze Plugin Code
\end{itemize}

Die meisten anderen Dateien sind leichter zu erraten.

\section{Schreiben von Scripts}\label{scripte}
Dagmar d'Surreal hat eine Dokumentation f�r das Schreiben von Scripten geschrieben (in xchatdox2.html).

\section{Schreiben von Plugins}
Es sollte ein Vorlagenmodul im Sample-Verzeichnis vorhanden sein, das einen generellen �berblick gibt, um ein Modul zu schreiben.


Als erstes solltest Du \texttt{\#define USE\_PLUGIN} benutzen, bevor Du andere \texttt{\#includes} schreibst. Du solltest au�erdem \texttt{xchat.h} und \texttt{plugin.h} aus dem Haupt-XChat Verzeichnis benutzen.
Jedes Modul sollte eine Funktion exportieren, die als \texttt{module\_init} benannt wird. Die Versionsnummer (ein int), ein Zeiger zur Modulstruktur f�r Dein Modul und ein Zeiger der derzeitigen Sitzung werden �bergeben.
Sie wirft ein int zur�ck:
\begin{itemize}
\item[0] = Erfolg
\item[1] = fehlgeschlagen
\end{itemize}

Der Name und der Beschreibungsteil der Struktur sollten mit Zeichenketten ausgef�llt werden.


Du solltest die Versionsnummer, welche Du denkst, die es gerade ist, �berpr�fen, bevor Du irgendwelche Referenzen aufbaust. Die derzeitige Versionsnummer wird in \texttt{plugin.h} als \texttt{MODULE\_IFACE\_VER} definiert.


Der eigentliche Haken in XChat ist das Signal. An einigen Stellen im Code wird ein Signal gesendet.
\dots

