\chapter{Fenster}\index{Fenster}
Neben dem Hauptfenster lassen sich noch weitere Fenster �ffnen. Sie dienen
dazu die Funktionen des IRC besser und angenehmer zu gestalten.

\section{Server-Liste}

\section{Kanallisten-Fenster}\label{sec:kanallisten}\index{Kanallisten-Fenster}
Dieses Fenster erlaubt es Dir, alle Kan�le auf
einem Server anzeigen zu lassen. Die Kan�le werden mit der Ber�cksichtigung auf
die gegebenen \emph{\textbf{Minimum Users}} gefiltert. Mit
\emph{\textbf{Refresh the list}} wird die Liste gel�scht und die Suche wird
neu gestartet. Mit \emph{\textbf{Save the list}} kann man die Liste in eine
Datei schreiben lassen, w�hrend man mit \emph{\textbf{Join Channel}} einen
Kanal betritt.


Denke daran, dass es tausende von Kan�len geben kann und mit dieser Suche Deine Bandbreite ganz sch�n beansprucht werden kann. Der einzige Weg eine durchlaufende Liste zu stoppen, ist - die Verbindung zu trennen.

\section{DCC Send Window und DCC Receive Window}\label{sec:dccsend}\index{DCC
  Send Window und DCC Receive Window}
Diese Fenster zeigen den Status von allen laufenden DCC Sendungen und Empf�ngern.
\begin{description}
  \item[Status] zeigt den Status der Datei\index{Datei ! Status} 
  \item[File] zeigt den Dateinamen\index{Datei ! Name}
  \item[Size] zeigt die Gr��e in Bytes\index{Datei ! Gr�sse}
  
  \item[Position] gibt die derzeitigen gesendeten bzw. empfangenen Bytes
  an\index{Datei ! Position}

  \item[Ack] (nur in Send) gibt die Anzahl der best�tigten Bytes
  an\index{Datei ! Ack}

  \item[CPS] gibt die Anzahl der Bytes an, die gesendet bzw. empfangen
  wurden\index{Datei ! CPS}

  \item[From] gibt den Nicknamen an die zu sendende bzw. empfangende Person
  an.\index{Datei ! From}

\end{description} 
Solltest Du GNOME benutzen, wird Dir noch der MIME-Typ der Datei angezeigt.\\
Nur in dem \emph{Receive Window} gibt es \textbf{Accept} und \textbf{Resume} Kn�pfe. \textbf{Accept} akzeptiert eine angebotene Datei, w�hrend \textbf{Resume} das gleiche macht, blo� mit dem Unterschied, dass es 
einen abgebrochenen Download wieder aufnimmt.


Der Text der Bestandteile im DCC-Fenster ist jetzt farbig mit dem Status der
�bertragung.

\section{DCC Chat Fenster}\label{sec:dccchat}\index{DCC Chat Fenster}
Das DCC Chat Fenster listet alle derzeitigen DCC Chat Sitzungen.\emph{\textbf{To/From}} gibt den Spitznamen des Gegen�bers.\emph{\textbf{Recv}} gibt die Anzahl der Bytes, die durch den DCC Link �bertragen wurden und \emph{\textbf{Send}} gibt die Anzahl der Bytes, die gesendet wurden an. \emph{\textbf{StartTime}} gibt die Zeit an, an der der Link aufgenommen wurde.

\section{Rohes Logbuch Fenster}\label{sec:rawlog}\index{Rohes Logbuch Fenster}
Das Rohe Logbuch Fenster listet die rohen Daten, die durch den Server gesendet und empfangen wurden, auf. Jede neue Zeile mit Daten beginnt mit ``$<<$'' oder ``$>>$''. Ein ``$<<$'' steht f�r den Rest der Zeile (nach dem Leerraum) f�r Daten von XChat zum Server. ``$>>$'' steht f�r den Rest der Zeile (nach dem Leerraum) f�r Daten vom Server zu XChat. Man kann auch durch Bet�tigen von \textbf{ALT-S} das rohe Log abspeichern. Man wird dann nach einem Dateinamen gefragt.

\section{URL Grabber}\label{sec:urlgrab}\index{URL Grabber}
Wenn eine URL (Uniform Resource Locator) in einem Fenster erscheint, wird diese im URL Grabber Fenster angezeigt. Der \textbf{Clear} Knopf l�scht die Liste. Der \textbf{Lynx} oder \textbf{Netscape} startet Lynx oder Netscape mit der ausgew�hlten URL aus der Liste.

\section{Benachrichtigungsliste}\label{sec:benachrichtigungsliste}\index{Benachrichtigungsliste}
Die Benachrichtigungsliste benutzt das \texttt{ISON} Kommando, um Freunde im IRC zu finden. Du kannst auch das \texttt{/notify} Kommando benutzen, um Leute hinzuzuf�gen oder zu entfernen. Die Benachrichtigungsliste zeigt dann, welche online sind und welchen Server sie benutzen. Der ``Remove'' Knopf l�scht den gerade ausgew�hlten 
Spitznamen von der Benachrichtigungsliste.


\section{Ignore Fenster}\label{sec:ignore}\index{Ignore Fenster}
Dieses Fenster kontrolliert die XChat Ignorieren-Funktion. Es (wie der Name schon vermuten mag) l�sst Dir Regeln aufstellen, um Nachrichten von Leuten zu ignorieren. Diese Regeln bestehen aus der Hostmaske und den Regeln was ignoriert werden soll. Die Maske ist im Format wie\\
\texttt{Spitzname!WirklicherName@host}. Also trifft \texttt{*!*@*.aol.com} auf jeden AOL Benutzer zu und \texttt{LameNick!*@*} w�rde auf jeden zutreffen, der mit \textbf{LameNick} anf�ngt. Die Leiste der Kn�pfe in der Mitte geben die Maske an, was ignoriert werden soll:

\begin{itemize}
\item[CTCP] - alle CTCP Nachrichten (DCC Send, CTCP Ping usw.)
\item[Private] - alle privaten Nachrichten, die mit \texttt{/msg} abgesetzt wurden
\item[Channel] - alle Kanalnachrichten 
\item[Notice] - alle \texttt{/notice} Nachrichten 
\item[Invite] - alle \texttt{/invite} Nachrichten
\item[Unignore] - Invertiert die Maske, so dass z.B. \textbf{*!*@*.aol.com} verbannt, als ignoriert werden kann.
\end{itemize}
Das Textk�stchen am unteren Ende zeigt die Anzahl wie oft eine Nachricht geblockt wurde. Die Unignore Funktion kann auch aus der Kommandozeile erreicht werden:
\begin{quote}
\texttt{/ignore *!*@*.aol.com ALL /ignore myfriend!myfriend@*.aol.com ALL UNIGNORE (W�rde alle von AOL ignorieren, au�er myfriend).}
\end{quote}
